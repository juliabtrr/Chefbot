\documentclass[a4paper,12pt]{article}
\usepackage[utf8]{inputenc}
\usepackage[T1]{fontenc}
\usepackage[french]{babel}
\usepackage{graphicx}
\usepackage{geometry}
\usepackage{color}
\usepackage{setspace}
\usepackage{titling}
\usepackage{xcolor}
\usepackage{hyperref}
\usepackage{float}
\usepackage{eso-pic}

\geometry{left=2cm,right=2cm,top=2cm,bottom=2cm}
\pagestyle{empty}

\begin{document}

\newcommand{\borduregauche}{
\AddToShipoutPictureFG*{
\put(-35,420){\includegraphics[height=\paperheight]{Template Latex pour rapport de projet L1/Bordure.png}}
}
}

\borduregauche

\begin{titlepage}

\begin{center}
\vspace*{1cm}
\hspace*{-2cm}
\includegraphics[width=9.5cm]{Template Latex pour rapport de projet L1/logo_Paris_Nanterre_couleur_RVB.png}
\vspace{0.5cm}
\noindent

\begin{center}
\vspace{0.0cm}
\noindent
\begin{center}
{\Large Licence économie-gestion (CMI) – 2\textsuperscript{e} année}\\[0.8cm]

{\Huge \textbf{Rapport de projet informatique}}\\[2.5cm]
{\Huge \textit{Chefbot}}\\[1.9cm]

{\Large Projet réalisé de septembre 2025 au 6 janvier 2026}\\[1.5cm]

{\Large Membres du groupe :}\\[0.5cm]

\begin{center}
\begin{minipage}{0.3\textwidth}
\centering
\textit{Betour julia\\44001299}
\end{minipage}
\hfill
\begin{minipage}{0.3\textwidth}
\centering
\textit{Sacko Fatoumata\\43005103}
\end{minipage}
\end{center}

\vspace{1,5cm}
\textit{Année universitaire 2025–2026} \\
\href{}{Lien GitHub : https://github.com/juliabtrr/Chefbot} \\ \end{center}

\end{titlepage}

\newpage
\thispagestyle{empty} % pas de numéro sur cette page


\tableofcontents
\vspace{1cm}
\end{center}

\newpage

\section{Remerciements}

Nous tenons à remercier toutes les personnes, collègues, familles et proches. qui
ont cru en nous ainsi qu’en notre projet. Nous tenons également à remercier
monsieur Bouquet pour son aide et ses conseils qui nous ont permis d’améliorer
notre projet au maximum.


\clearpage
\section{Introduction}

Avec l’évolution des technologies numériques et l’essor de l’intelligence artificielle
générative, de nouvelles solutions émergent pour simplifier les tâches du quotidien. La
cuisine, bien qu’ essentielle, représente souvent une contrainte pour de nombreuses
personnes par manque de temps, d’inspiration ou de planification.
Dans ce contexte, notre projet informatique consiste à concevoir une application web
intégrant une intelligence artificielle générative capable d’assister les utilisateurs dans la
création de recettes de cuisine personnalisées. L’application, nommée ChefBot, permet
de générer automatiquement des recettes complètes à partir des ingrédients
disponibles, d’améliorer des recettes existantes, de proposer une liste d’ingrédients
manquants, de fournir des conseils culinaires adaptés et d’offrir une lecture audio des
recettes.
L’objectif principal de ChefBot est de proposer une interface simple, intuitive et
moderne permettant à l’utilisateur de transformer les ingrédients dont il dispose en une
recette réaliste, exploitable et personnalisée, tout en s’adaptant à ses préférences
alimentaires.
\clearpage
\section{Environnement de travail}
Dans un premier temps, nous avons réfléchi à l’environnement de développement le
plus adapté à la réalisation de notre projet ChefBot, qui repose sur une application web
moderne intégrant une intelligence artificielle générative. Il était essentiel de choisir
des outils à la fois performants, flexibles et adaptés au développement d’applications
web récentes.
Après analyse, nous avons opté pour Visual Studio Code, un éditeur de code largement
utilisé dans le développement web, offrant une grande compatibilité avec les
technologies que nous souhaitions utiliser. Cet environnement nous a permis de
travailler efficacement grâce à ses nombreuses extensions, son terminal intégré et sa
bonne gestion des projets JavaScript et TypeScript.
4
Le projet a été développé sur un PC sous Windows 11, en utilisant les langages
JavaScript et TypeScript, ainsi que le framework Next.js 14, basé sur React et
l’architecture App Router. Pour la gestion de l’interface et du design, nous avons utilisé
Tailwind CSS, qui permet de créer des interfaces modernes, responsives et lisibles de
manière rapide et structurée.
La logique côté serveur a été gérée à l’aide des API Routes de Next.js, qui assurent la
communication entre le front-end et l’API d’intelligence artificielle Google Gemini. Les
tests de l’application ont été réalisés principalement sur le navigateur Google Chrome,
afin de vérifier le bon fonctionnement de l’interface et des fonctionnalités.
Cette combinaison d’outils et de technologies nous a permis de développer une
application fluide, cohérente et facile à maintenir, tout en répondant aux exigences
techniques du projet ChefBot.

\clearpage

\section{Description du projet et objectifs}
\subsection{Objectif générale}
Avant de commencer le développement de ChefBot, nous avons défini l’objectif principal
du projet afin de guider l’ensemble de sa conception. L’objectif général de ChefBot est
d’assister l’utilisateur dans la création de recettes de cuisine personnalisées à l’aide
d’une intelligence artificielle générative.
L’application permet à l’utilisateur de saisir les ingrédients dont il dispose, accompagnés
de leurs quantités, afin de générer automatiquement une recette complète, réaliste et
exploitable. ChefBot prend également en compte différents modes alimentaires, tels que
le mode standard, healthy, riche en protéines ou végétarien, afin de s’adapter aux
besoins et préférences de chacun.
En plus de la génération de recettes, ChefBot propose des améliorations de plats
existants, une liste des ingrédients manquants à acheter ainsi que des conseils
culinaires adaptés. L’objectif est de se rapprocher d’un usage réel en cuisine, en offrant
une aide concrète, simple et intuitive, tout en optimisant l’organisation des repas et en
réduisant le manque d’inspiration culinaire.

\subsection{B. Public visée}
ChefBot s’adresse à un public large, mais vise en priorité les personnes recherchant une
solution simple et rapide pour cuisiner sans planification complexe. Le public ciblé est
principalement composé :
●
des étudiants, disposant de peu de temps et d’un budget limité ;
●
des personnes vivant seules, souhaitant optimiser leurs repas avec les
ingrédients disponibles ;
●
de toute personne désirant cuisiner facilement, sans connaissances culinaires
avancées.
Grâce à son interface claire et accessible, ChefBot est conçu pour être utilisé par tous
les profils, quel que soit le niveau en cuisine ou en informatique. L’objectif est de rendre
la préparation des repas plus simple, plus efficace et plus agréable au quotidien.

\subsection{C. Description du projet}
Le cœur de notre projet repose sur une interface interactive dans laquelle l’utilisateur
saisit les ingrédients dont il dispose, accompagnés de leurs quantités. Ces informations
sont ensuite envoyées à une route API interne développée avec Next.js, qui se charge de
formuler un prompt structuré destiné à l’intelligence artificielle. À partir de ces
données, l’IA génère une recette complète et adaptée aux éléments fournis par
l’utilisateur.
L’application permet également de sélectionner un mode alimentaire spécifique
(standard, healthy, riche en protéines ou végétarien), afin d’affiner la génération de la
recette. En complément, ChefBot propose une version améliorée de la recette générée,
une liste des ingrédients manquants à acheter ainsi que des conseils culinaires
personnalisés pour optimiser la préparation du plat.
Enfin, une fonctionnalité de lecture audio permet à l’utilisateur d’écouter la recette
étape par étape, facilitant ainsi son utilisation en situation réelle de cuisine. L’interface
de ChefBot se veut conviviale, moderne et simple d’utilisation, tout en restant accessible
sur différents navigateurs et supports. Le design épuré met en valeur les informations
essentielles et contribue à offrir une expérience fluide et agréable à l’utilisateur.

\clearpage

\section{Bibliothèques, outils et technologies}

\subsubsection*{A. Back END}
Pour la partie back-end du projet ChefBot, nous avons utilisé les API Routes de Next.js
14, qui permettent de créer des endpoints côté serveur et de gérer la communication
entre le front-end et l’intelligence artificielle. Ces routes assurent le traitement des
données envoyées par l’utilisateur et la transmission des résultats générés vers
l’interface.
L’interaction avec l’intelligence artificielle est réalisée grâce à la bibliothèque
@google/generative-ai, permettant de communiquer avec l’API Google Gemini. Cette
dernière est utilisée pour générer des recettes culinaires structurées à partir des
ingrédients fournis. Le back-end veille également à la structuration des réponses au
format JSON, garantissant ainsi une exploitation fiable et cohérente des données par le
front-end.

\subsubsection*{B. Front END}
Pour la partie front-end du projet ChefBot, nous avons utilisé le framework React,
intégré à Next.js 14 à travers l’architecture App Router. Ce choix nous a permis de
structurer l’application sous forme de composants réutilisables et de gérer efficacement
l’affichage dynamique des données. React facilite la mise à jour de l’interface en fonction
des interactions de l’utilisateur, ce qui est essentiel pour l’affichage des recettes
générées, des conseils culinaires et des différentes options proposées par l’application.
La mise en forme de l’interface utilisateur a été réalisée à l’aide de Tailwind CSS. Cette
bibliothèque utilitaire nous a permis de concevoir une interface moderne, responsive et
orientée vers la lisibilité. Grâce à Tailwind CSS, nous avons pu adapter l’affichage de
l’application à différents formats d’écran tout en conservant une cohérence visuelle et
une navigation fluide.
Enfin, nous avons intégré la Web Speech API afin d’ajouter une fonctionnalité de lecture
audio des recettes. Cette API permet de convertir le texte généré par l’intelligence
artificielle en audio, offrant ainsi à l’utilisateur la possibilité d’écouter les étapes de la
recette. Des contrôles tels que la lecture, la pause, la reprise et l’arrêt ont été mis en
place afin d’améliorer l’expérience utilisateur, notamment lors de l’utilisation de
l’application en situation réelle de cuisine.

\subsubsection*{C. API utilisée}
Dans le cadre du projet ChefBot, nous avons utilisé la Google Gemini API afin d’intégrer
une intelligence artificielle générative au sein de l’application. Cette API permet
d’interagir avec des modèles de langage avancés capables de comprendre des
instructions complexes et de générer du contenu textuel structuré.
Le modèle utilisé est gemini-2.0-flash, choisi pour sa rapidité d’exécution et sa capacité
à produire des réponses cohérentes et pertinentes en un temps réduit. Ce modèle est
particulièrement adapté à un usage interactif, où l’utilisateur attend une génération de
contenu quasi immédiate après la saisie de ses informations.
Le rôle principal de cette API est la génération de contenu culinaire structuré. À partir
des ingrédients et des quantités fournis par l’utilisateur, l’intelligence artificielle est
capable de proposer une recette complète, incluant les étapes de préparation, des
conseils culinaires, ainsi qu’une liste des ingrédients manquants. L’utilisation de l’API
Gemini constitue ainsi le cœur intelligent de ChefBot et permet de rendre l’application
réellement fonctionnelle et personnalisée.




\section{ Fonctionnement de l’intelligence artificielles et
technologies}
L’intelligence artificielle intégrée au projet ChefBot repose sur un modèle de type LLM
(Large Language Model), capable de générer du contenu textuel structuré à partir
d’instructions précises.
Le fonctionnement de l’IA s’effectue selon un processus en plusieurs étapes. Tout
d’abord, l’utilisateur saisit les ingrédients dont il dispose ainsi que leurs quantités via
l’interface de l’application. Ces informations sont ensuite transmises à une route API
interne, où un prompt structuré est automatiquement généré côté serveur. Ce prompt
est envoyé à l’API Gemini, qui traite la demande et renvoie une réponse respectant un
format JSON strict. Enfin, le front-end exploite ces données afin d’afficher la recette
générée, les étapes de préparation et les conseils culinaires associés.
L’intelligence artificielle n’enregistre aucune donnée personnelle ou information saisie
par l’utilisateur, garantissant ainsi le respect de la confidentialité et de la vie privée.

\section{ Utilisation de l’IA dans la réalisation du projet}
L’intelligence artificielle a également été utilisée comme outil d’assistance au
développement tout au long de la réalisation du projet ChefBot. Elle a notamment
permis de faciliter la génération de composants React, d’aider à la résolution d’erreurs
techniques rencontrées lors du développement et d’optimiser la logique générale de
l’application.
De plus, l’IA a été utilisée pour améliorer la conception des prompts culinaires, afin
d’obtenir des réponses plus pertinentes et mieux structurées. Elle a également
contribué à la structuration des réponses au format JSON, garantissant une meilleure
fiabilité dans l’échange de données entre le back-end et le front-end.

\section{ Travail réalisé}
Au cours de ce projet, plusieurs fonctionnalités ont été mises en œuvre afin de répondre
aux objectifs définis. L’application permet la génération automatique de recettes à
partir des ingrédients disponibles, tout en prenant en compte les quantités saisies par
l’utilisateur. Différents modes alimentaires ont été intégrés, notamment les modes
standard, healthy/light, high protein et végétarien, afin d’adapter les recettes aux
préférences et aux contraintes alimentaires.
L’application propose également une version améliorée des recettes, une liste des
ingrédients manquants à acheter, ainsi que des conseils culinaires personnalisés. Une
fonctionnalité de lecture audio des recettes a été ajoutée, permettant à l’utilisateur de
lancer, mettre en pause, reprendre ou arrêter la lecture. Enfin, une interface moderne
et responsive a été développée afin d’assurer une utilisation confortable sur différents
supports.
Certaines fonctionnalités n’ont toutefois pas été réalisées dans le cadre de ce projet. Il
s’agit notamment de l’authentification des utilisateurs, de la mise en place d’un
historique persistant des recettes, ainsi que du stockage des données en base de
données.




\section{Difficultés rencontrées}
Au cours du développement du projet, plusieurs difficultés techniques ont été
rencontrées. La gestion des quotas de l’API Google Gemini a nécessité une
attention particulière afin d’éviter les limitations lors des phases de test. La
fiabilité du format JSON retourné par l’intelligence artificielle a également posé
problème, ce qui a conduit à renforcer la structuration des réponses générées.
Par ailleurs, la gestion de la lecture audio, notamment les fonctionnalités de
pause et d’arrêt, a demandé des ajustements afin d’assurer un fonctionnement
stable. Enfin, la stabilisation du flux de navigation entre les différentes pages
de l’application a nécessité plusieurs corrections. L’ensemble de ces difficultés a
été progressivement résolu grâce à des tests réguliers et à une structuration
rigoureuse du code.
\section{Mise en fonctionnement de l’application}
Afin de mettre en fonctionnement l’application ChefBot, certains prérequis
techniques sont nécessaires. Il est indispensable de disposer de Node.js (version
18 ou supérieure) ainsi que d’une clé API Google Gemini valide.
L’installation des dépendances du projet s’effectue à l’aide de la commande npm
install. Une fois l’installation terminée, une phase de configuration est
requise. Pour cela, il convient de créer un fichier .env.local à la racine du projet
et d’y renseigner la clé API Gemini selon le format suivant :
GEMINI_API_KEY=VOTRE_CLE_API.
Enfin, l’application peut être lancée à l’aide de la commande npm run fois le serveur démarré, l’accès à l’application se fait via l’adresse suivante :
dev. Une
http:/ /localhost:3000.

\section{ Conclusion et perspectives}
ChefBot est une application qui illustre l’intégration efficace d’une intelligence
artificielle générative au sein d’un projet web moderne. Ce projet a permis de mettre en
œuvre des technologies actuelles telles que Next.js, React et l’API Google Gemini, tout
en proposant une solution utile et concrète répondant à un besoin du quotidien.
L’application offre une expérience utilisateur intuitive et démontre le potentiel de l’IA
dans l’assistance culinaire personnalisée.
Plusieurs axes d’amélioration peuvent être envisagés pour faire évoluer l’application. Il
serait notamment possible d’ajouter un historique des recettes, une fonctionnalité
d’export au format PDF, ainsi qu’une estimation nutritionnelle des plats générés.
L’intégration de la génération d’images culinaires permettrait d’enrichir davantage
l’expérience utilisateur, tandis qu’un mode hors ligne basé sur des données simulées
pourrait améliorer l’accessibilité de l’application.

\section{ Bibliographie}
•Documentation Next.js

•Documentation Google Gemini API

•Documentation Tailwind CSS

\section{ Webographie}
•https:/ /nextjs.org

•https:/ /ai.google.dev

•https:/ /tailwindcss.com

\section{ Annexes}




\subsection{Cahier des charges}
●Le cahier des charges définit les objectifs et les fonctionnalités principales de l’application ChefBot. Il inclut :

La description des fonctionnalités attendues, détaillant le rôle de chaque
module de l’application.

●Les maquettes de l’interface, présentant la structure des différentes pages et
l’agencement des éléments pour assurer une expérience utilisateur intuitive et
cohérente.



\subsection{Exemples d’exécution et de prompt}
exemple discussion prompt avec l’intelligence artificielle :
Plusieurs exemples illustrent le fonctionnement concret de l’application :


\begin{figure}[H]
\centering
\begin{minipage}{0.48\textwidth}
\centering
\includegraphics[width=\textwidth]{Capture d’écran 2026-01-05 à 12.50.17.png}
\caption{l'utilisateur rentre ici les ingrédients et le régime alimentaire souhaité}
\end{minipage}
\hfill
\begin{minipage}{0.48\textwidth}
\centering
\includegraphics[width=\textwidth]{Capture d’écran 2026-01-05 à 12.52.24.png}
\caption{la recette est générée }
\end{minipage}
\end{figure}

\begin{figure}[H]
\centering
\begin{minipage}{0.48\textwidth}
\centering
\includegraphics[width=\textwidth]{Capture d’écran 2026-01-05 à 12.47.30.png}
\caption{possibilité de se faire dictée la recette et de mettre en pause}
\end{minipage}
\hfill
\begin{minipage}{0.48\textwidth}
\centering
\includegraphics[width=\textwidth]{IMG_3917.PNG}
\caption{prompt}
\end{minipage}
\end{figure}

●Génération de recette : création automatique d’une recette à partir des
ingrédients saisis par l’utilisateur.
●Amélioration de recette : proposition de variantes ou d’optimisations pour
enrichir la recette initiale.

●Liste des ingrédients à acheter : identification des ingrédients manquants
nécessaires à la réalisation complète de la recette.

\subsection{Manuel utilisateur}
Le manuel utilisateur présente les étapes pour exploiter pleinement ChefBot :
1. Saisir les ingrédients disponibles ainsi que leurs quantités.
2. Choisir le mode alimentaire correspondant aux préférences ou contraintes
personnelles.
3. Générer la recette automatiquement.
4. Écouter la recette via la fonctionnalité audio ou appliquer les suggestions
d’amélioration proposées par l’application.



\end{document}

